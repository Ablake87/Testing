\documentclass[10pt,twocolumn]{article}
	
\usepackage{myfontstyle}
\usepackage{mypackages}
\usepackage{mymacros}
\usepackage{mycommands}

\begin{document}
\thispagestyle{fancy1}

%%% Title and Abstract------------------------
\twocolumn[
\begin{center}
	\hrule
	\vspace{3pt}
	% Title:
	{\sffamily\bfseries\Large
		Report for Laboratory Two: Voltage Dividers
	} \\
	{\color{gray}
		\vspace{3pt}
		\hrule
		\vspace{3pt}
	}
	{
		\hspace*{\fill}
		Austin Piper
		\hspace*{\fill}
		Alex Blakely
		\hspace*{\fill}
		Ahmed Irfan
		\hspace*{\fill}
%		Fourth Author    % uncomment these two lines if there's a fourth author
%		\hspace*{\fill}
	}\\
	\vspace{3pt}
	{\itshape
		\hspace*{\fill}
		Department of Mechanical Engineering, Saint Martin's University
		\hspace*{\fill} \\
		\hspace*{\fill}
		ME/EE 316---Mechatronics \& Measurements Laboratory
		\hspace*{\fill}
	}\\
	\vspace{3pt}
	{
		\hspace*{\fill}
		\today{} % today's date ... can type manually instead
		\hspace*{\fill}
	}
	\vspace{3pt}
	{\color{gray}\hrule}
%	\vspace{2pt}
\end{center}
% Abstract:
\begin{adjustwidth}{1.5in}{1.5in}
{\small
\noindent\textbf{Abstract.} \hspace{1em}
	Applying a 10V DC power supply to a voltage divider circuit with two resistors in series, the voltage across both resistors remains constant while the voltage across a single resistor changes depending on the resistance. Using a myRIO  configured with the labVIEW software to change our input voltage and measure the voltage source and the resistor voltage. When connecting an arbitrary function generator to an oscilloscope, a variety of wave functions at 5Vpp had a period of 2.5ms.
}
\end{adjustwidth}
\vspace{9pt}
\hrule
\vspace{1\baselineskip}
]

%%% Body -------------------------


\section{Introduction} 
\label{sec:introduction}

 ``Introduce what your question is. Explain why someone should find this interesting. Summarize what is currently known about the question. Introduce a little of what you found and how you found it. You should explain any ideas or techniques that are necessary for someone to understand your results section.''

Typically, you will be asked to include in your report a theoretical prediction for comparison with your experimental results. Typically, this section is a good place for presenting that. You needn't always include every aspect of your derivation, but you should "sketch out" the process, hitting the "highlight" ideas an equations along the way. Typically there's no need to, say, plot the results here, since you'll be plotting them again alongside the data in the Results section (\autoref{sec:results}). 

\section{Materials and Methods}

	This lab has three parts. In the first part of the lab two resistors were placed in series on a bread board connected to a DC power supply set to 10V. Using a multimeter measure the voltage across each resistor and both resistors. Then replacing the second resistor and taking measurements again for all four different resistors.\autoref{fig:circuit}
	
	
	For part two of this lab a myRIO configured with the labVIEW software was be used as the power supply and measurement tool, on the same voltage divider circuit. The myRIO was connected to measure the voltage across both resistors and the voltage across the second resistor. Using labVIEW an analog output and input were made to recieve data from the myRIO, as well as a voltage vs time chart window. Starting at 0V and working up to 10V, in increments of 1V, the voltage source and the resistor voltage is displayed in labVIEW and recorded.Again replacing the second resistor with each of the different resistors and repeating measurements.\autoref{fig:myrio} 
	

	In part three of this lab an arbitrary function generator was connected to an oscilloscope. While producing a sine wave with the function generator set to 5Vpp, then by adjusting the oscilloscope settings a stable wave was found and the peak to peak amplitude and period were estimated.\autoref{fig:stable}
	
\begin{figure}
	\centering
	\includegraphics[width=.9\linewidth]{figures/vdc.pdf}
	\caption{Voltage divider circuit}
	\label{fig:circuit}
\end{figure}

\begin{figure}
	\centering
	\includegraphics[width=.9\linewidth]{figures/myr.pdf}
	\caption{myRIO Ciruit}
	\label{fig:myrio}
\end{figure}

\begin{figure}
	\centering
	\includegraphics[width=.9\linewidth]{figures/Oscilloscope.PNG}
	\caption{Oscilloscope}
	\label{fig:stable}
\end{figure}


\section{Results}

 In the first section of the lab the results showed that as the total resistance changed for the circuit the total voltage across the circuit was not changing. The voltage across the second resistor changed when the resistance changed. \autoref{tab:Tab1}
 
 	After connecting the myRIO to the circuit and proceeding through the different input voltages we found that, again, the voltage through the entire circuit was constant no matter the total resistance and the voltage through the resistor depended on the amount of resistance.\autoref{tab:Tab2}
 	
 	When observing the stable wave made by the function generator in the oscilloscope it was estimated that the peak-peak amplitude was a 5V with a period of 2.5ms. Even when changing the shape of the wave the period and amplitude stayed the same.  
 	  
\section{Equations}


\begin{align}
V=iR


\bm{VR1} = \bm{Vs} 
    	\left(
        	\frac{R1}{R1+R2}
        \right), 
\end{align}
\section{Discussion}
Above all else, this lab demonstrated the efficacy of what we learned in class lecture concerning the voltage divider equation and it made our group more familiar with the technology we will be continuously using throughout the semester and possibly in future careers. We have learned exactly what happens in a circuit that contained a voltage divider (two resistors in series with an outpot voltage that is a fraction of the input). During the initial portion of the lab when measuring voltage drops across the whole circuit vs. the resistors we would see consistent drops of 60%-30% during the latter measurements compared to the former which saw almost no voltage drop at all. 

When using the myRIO software and pushing different source voltages through the various resistors the results were predictable and similar to the first part of the lab, with the voltages across the resistors being a fraction of the source voltage. Resistors 2 and 3 were at about a 50% voltage loss, and resistors 4 and 5 saw about a 40% drop. This further reinforces what we had learned about the properties of a voltage divider circuit. Speaking to other groups in the lab they saw more or less the same results.

In our last section the group had trouble setting up the oscilloscope to produce a solid sine wave shape. After spending an inordinate amount of time we were able to achieve success and estimated, as stated above in our results section, that the peak amplitude was 5V. We shifted the wave to different shapes like a square wave and this did not change. 

\begin{figure}
	\centering
	\includegraphics[width=.9\linewidth]{figures/V vs R.PNG}
	\caption{}
	\label{fig:graph}
\end{figure}

\begin{figure}
	\centering
	\includegraphics[width=.9\linewidth]{figures/Vin vs Vout.PNG}
	\caption{Vin vs Vout(Measured and Theoretical)}
	\label{fig:vin/out}
\end{figure}

\section{Author Contributions}



 
\begin{table}
	\begin{tabularx}{1\linewidth}{ lXXXXX|cXXX }
		\hline
		 & \textbf{R1} & \textbf{R2} & \textbf{R3} & \textbf{R4} & \textbf{R5}\\
		\hline
		Ri(MΩ) & 1.4667 & 1.4679 & 2.2 & 3.239 & 4.712 \\
		\cline{1-6}
		Vs(V) & 10.093 & 10.093 & 10.091 & 10.093 & 10.097 \\
		\cline{1-6}
		Vr(V) & 4.691 & 4.708 & 5.547 & 6.321 & 6.936 \\
		\hline
	\end{tabularx}
	\caption{Multimeter measurments.}
	\label{tab:Tab1}
\end{table}

\begin{table}
	\begin{tabularx}{1\linewidth}{ lXXXXX|cXXXXXXXXXXXXXXXXXXXX }
		\hline
		 & \textbf{nom. Vs} & \textbf{R2} & \textbf{R3} & \textbf{R4} & \textbf{R5}\\
		\hline
		Vs(V) & 0 & 0.006 & 0.006 & 0.006 & 0.006 \\
		Vr(V) & 0 & 0.002 & 0.003 & 0.003 & 0.004 \\
		\cline{1-6}
		Vs(V) & 1 & 1.0101 & 1.010 & 1.010 & 1.010 \\
		Vr(V) & 1 & 0.471 & 0.555 & 0.632 & 0.693 \\
		\cline{1-6}
		Vs(V) & 2 & 2.015 & 2.015 & 2.015 & 2.015 \\
		Vr(V) & 2 & 0.940 & 1.107 & 1.241 & 1.384 \\
		\cline{1-6}
		Vs(V) & 3 & 3.013 & 3.013 & 3.013 & 3.013 \\
		Vr(V) & 3 & 1.406 & 1.656 & 1.887 & 2.070 \\
		\cline{1-6}
		Vs(V) & 4 & 4.018 & 4.018 & 4.018 & 4.018 \\
		Vr(V) & 4 & 1.875 & 2.209 & 2.516 & 2.760 \\
		\cline{1-6}
		Vs(V) & 5 & 5.023 & 5.023 & 5.023 & 5.023 \\
		Vr(V) & 5 & 2.344 & 2.761 & 3.145 & 3.450 \\
		\cline{1-6}
		Vs(V) & 6 & 6.027 & 6.027 & 6.027 & 6.027 \\
		Vr(V) & 6 & 2.813 & 3.313 & 3.774 & 4.140 \\
		\cline{1-6}
		Vs(V) & 7 & 7.032 & 7.032 & 7.032 & 7.032 \\
		Vr(V) & 7 & 3.282 & 3.866 & 4.403 & 4.830 \\
		\cline{1-6}
		Vs(V) & 8 & 8.030 & 8.030 & 8.030 & 8.030 \\
		Vr(V) & 8 & 3.748 & 4.415 & 5.028 & 5.516 \\
		\cline{1-6}
		Vs(V) & 9 & 9.035 & 9.035 & 9.035 & 9.035 \\
		Vr(V) & 9 & 4.218 & 4.967 & 5.658 & 6.207 \\
		\cline{1-6}
		Vs(V) & 10 & 10.011 & 10.011 & 10.011 & 10.011 \\
		Vr(V) & 10 & 4.673 & 5.504 & 6.269 & 6.877 \\
		\hline
	\end{tabularx}
	\caption{myRIO w/ labVIEW read outs.}
	\label{tab:Tab2}
\end{table}

\end{document}  
